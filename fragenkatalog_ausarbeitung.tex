\documentclass[a4paper,10pt]{article}

\usepackage[utf8]{inputenc}
\usepackage{amssymb}
\usepackage{amsmath}
%\usepackage{textcomp}
%\usepackage{math}
%\usepackage{tcolorbox}
\usepackage{collectbox}

\title{Statistik Fragenkatalog - Ausarbeitung}
\author{}
\date{\today}

\pdfinfo{%
  /Title    (Statistik Fragenkatalog - Ausarbeitung)
  %/Author   ()
  %/Creator  ()
  %/Producer ()
  %/Subject  ()
  %/Keywords ()
}

\begin{document}
  \maketitle
  \newpage
  \tableofcontents{}
  \newpage
%%%%%%%%%%%%%%%%%%%%%%%%%%%%%%%%%%%%%%%%%%%%%%%%%%%%%%%%%%%%%%%%%%%%%%%%%%%%%%%%%%%%%%%%%%%%%%%%%%%%%%%%%%%%%%%%%%%%%%%%%%%%%%%%%%%%%%%%%%%%%%%%%%%
\section{Deskriptive Statistik}
\subsection{Berechne für eine gegebene Stichprobe zu den Klassengrenzen ... alle relativen Häufigkeiten und zeichne ein skaliertes Histogramm mit relativen Häufigkeiten, wobei der Flächeninhalt der Balken den Häufigkeiten entsprechen soll.}

\subsection{Gegeben ist eine Häufigkeitstabelle. Berechne das arithmetische Mittel, die Standardabweichung, den Median, das $n$. Quartil und den Modus.}

\subsection{Wie hängt das empirische Quantil mit der empirischen Verteilungsfunktion zusammen?}

\newpage
%%%%%%%%%%%%%%%%%%%%%%%%%%%%%%%%%%%%%%%%%%%%%%%%%%%%%%%%%%%%%%%%%%%%%%%%%%%%%%%%%%%%%%%%%%%%%%%%%%%%%%%%%%%%%%%%%%%%%%%%%%%%%%%%%%%%%%%%%%%%%%%%%%%
\section{Korrelation und Regression}
\subsection{Berechne aus einer zweidimensionalen Stichprobe den Korrelationskoeffizienten und die Regressionsgerade $(a, b)$. Zeichne den Scatterplot und dort die Regressionsgerade ein.}

\subsection{Was ist der Unterschied zwischen linearer Regression und einem linearen Regressionsmodell?}

\subsection{Zeige: Die Lösung einer linearen Regression ergibt sich aus der Lösung des linearen Gleichungssystems $C a = b$, wobei $a$ der Vektor der $m$ Parameter $a_1 , a_2 , . . .$ ist, $C$ eine $m × m$ Matrix und $b$ ein $m$-Vektor ist mit\newline
$C_{k,l}=\sum\limits_{i=1}^n f_k (x_i)f_l (x_i)$, $b_k = \sum\limits_{i=1}^n y_i f_k (x_i)$.}

\newpage
%%%%%%%%%%%%%%%%%%%%%%%%%%%%%%%%%%%%%%%%%%%%%%%%%%%%%%%%%%%%%%%%%%%%%%%%%%%%%%%%%%%%%%%%%%%%%%%%%%%%%%%%%%%%%%%%%%%%%%%%%%%%%%%%%%%%%%%%%%%%%%%%%%%
\section{Ereignis- und Wahrscheinlichkeitsraum}
\subsection{Zeige, dass ($\Omega, \Sigma$) = (\{1, 2, 3, 4\}, \{$\emptyset$, \{1, 2\}, \{3, 4\}, \{1, 2, 3, 4\}\}) ein Ereignisraum ist.}

\subsection{Für ($\Omega, \Sigma$) wie in 3.1 und $P$ (\{1, 2\}) = 0.3, vervollständige $P$, so dass ($\Omega, \Sigma, P$) ein Wahrscheinlichkeitsraum ist.}

\subsection{Beweise den Additionssatz.}

\newpage
%%%%%%%%%%%%%%%%%%%%%%%%%%%%%%%%%%%%%%%%%%%%%%%%%%%%%%%%%%%%%%%%%%%%%%%%%%%%%%%%%%%%%%%%%%%%%%%%%%%%%%%%%%%%%%%%%%%%%%%%%%%%%%%%%%%%%%%%%%%%%%%%%%%
\section{Kombinatorik (Blatt 04)} %TODO
Siehe PS-Beispiele.
\subsection{}

\subsection{}

\subsection{}

\newpage
%%%%%%%%%%%%%%%%%%%%%%%%%%%%%%%%%%%%%%%%%%%%%%%%%%%%%%%%%%%%%%%%%%%%%%%%%%%%%%%%%%%%%%%%%%%%%%%%%%%%%%%%%%%%%%%%%%%%%%%%%%%%%%%%%%%%%%%%%%%%%%%%%%%
\section{Bedingte Wahrscheinlichkeit}
\subsection{Beispiel zu totaler Wahrscheinlichkeit und Entscheidungsbaum (ähnlich zu Glühlampenkartons aus PS).}

\subsection{Beispiel zu Bayes (Blatt 05).}

\subsection{Formuliere und beweise den Satz von Bayes für Bedingung/Gegenbedingung $B, \bar{B}$.}

\newpage
%%%%%%%%%%%%%%%%%%%%%%%%%%%%%%%%%%%%%%%%%%%%%%%%%%%%%%%%%%%%%%%%%%%%%%%%%%%%%%%%%%%%%%%%%%%%%%%%%%%%%%%%%%%%%%%%%%%%%%%%%%%%%%%%%%%%%%%%%%%%%%%%%%%
\section{Zufallsvariablen}
\subsection{Erwartungswert und Varianz einer konkreten (neuen aber einfachen) diskreten oder stetigen Verteilung ausrechnen.}

\subsection{Definiere die Binomial-/geometrische Verteilung und leite Erwartungswert und Varianz her.}

\subsection{Erwartungswert herleiten für Poissonverteilung $f_X (k)=\frac{\lambda^k}{k!}e^{-\lambda}$.}

\subsection{Erwartungswert herleiten für Normalverteilung. Hinweis: zuerst Dichtefunktion differenzieren.}

\subsection{Definiere die Exponentialverteilung. Leite Verteilungsfunktion und Erwartungswert her.}

\subsection{Beispiel zur Poissonapproximation.}

\subsection{Beispiel zur Normalapproximation.}

\subsection{Definiere die Student-$t /\chi^2$ /F-Verteilung. Welche Parameter besitzt die Ver-
teilung? Wo wird diese Verteilung verwendet?}

\subsection{Beispiel ähnlich zu: Widerstände aus verschiedenen Schachteln ...}

\subsection{Definiere die Kovarianz zweier Zufallsvariablen. Für $X$ und $Y$ unabhängig mit der gleichen Verteilung, zeige: V($X + Y$) = 2 V($X$), aber V(2$X$) = 4 V($X$).}

\subsection{$X$ und $Y$ unabhängig mit selber spezieller einfacher Dichtefunktion. Berechne $f_{X + Y}$.}

\newpage
%%%%%%%%%%%%%%%%%%%%%%%%%%%%%%%%%%%%%%%%%%%%%%%%%%%%%%%%%%%%%%%%%%%%%%%%%%%%%%%%%%%%%%%%%%%%%%%%%%%%%%%%%%%%%%%%%%%%%%%%%%%%%%%%%%%%%%%%%%%%%%%%%%%
\section{Zentraler Grenzwertsatz}

\subsection{Zeige: Wenn $X \sim N_{0,1}$ , dann ist $\varphi_X = e^{- \frac{\omega^2}{2}}$. Vorgehensweise: Dichtefunktion ableiten, dann zeigen, dass $e^{- \frac{\omega^2}{2}}$ die dabei entstehende Differentialgleichung erfüllt.}

%irgendwie fehlerhaft,liefert aber das gewünschte
\subsection{Zeige: Es sei $Y_n:= \frac{1}{\sqrt{n}} (X_1 + X_2 + . . . + X_n)$ mit $X_k \sim N_{0,1}$ und unabhängig. Dann gilt $\varphi_{Y_{n}}(\omega)\xrightarrow{n\rightarrow \infty} e^{- \frac{\omega^2}{2}}$. Vorgangsweise: Zeige $\varphi_{Y_{n}}(\omega) = \varphi^{n}_{\frac{X}{\sqrt{x}}}$, setze die Exponentialreihe ein, behalte nur die ersten drei Glieder, erkläre, warum die anderen zu vernachlässigen sind, lasse dann unter Verwendung von $(1 + \frac{a}{n})^n \xrightarrow{n\rightarrow \infty} e^a$ das $n$ nach unendlich gehen.}

\newpage
%%%%%%%%%%%%%%%%%%%%%%%%%%%%%%%%%%%%%%%%%%%%%%%%%%%%%%%%%%%%%%%%%%%%%%%%%%%%%%%%%%%%%%%%%%%%%%%%%%%%%%%%%%%%%%%%%%%%%%%%%%%%%%%%%%%%%%%%%%%%%%%%%%%
\section{Schätzer}
\subsection{Schätzer entwickeln für spezielle einfache Verteilung mit Maximum Likelihood- oder Momentenmethode.}

\subsection{Speziellen Schätzer auf Erwartungstreuheit überprüfen.}

\subsection{Zeige, dass $s^2$ ein erwartungstreuer Schätzer für $V(X)$ ist.}

\newpage
%%%%%%%%%%%%%%%%%%%%%%%%%%%%%%%%%%%%%%%%%%%%%%%%%%%%%%%%%%%%%%%%%%%%%%%%%%%%%%%%%%%%%%%%%%%%%%%%%%%%%%%%%%%%%%%%%%%%%%%%%%%%%%%%%%%%%%%%%%%%%%%%%%%
\section{Konfidenzintervalle}
Die Formeln für die Konfidenzintervalle stehen auf dem Angabeblatt.

\subsection{Für bestimme Stichprobe einer Normalverteilung Konfidenzintervall für $\mu$ und/oder $\sigma$ ausrechnen, wobei $\sigma$ bekannt/unbekannt sein kann. Siehe PS-Beispiele.}

\subsection{Konfidenzintervall für bestimmte Stichprobe eines Bernoulli-Experiments ausrechnen. Siehe PS-Beispiele.}

\subsection{Zeige: $\frac{n-1}{\sigma^2} s^2 \sim \chi^{2}_{n-1}$}

\subsection{Zeige: $\frac{\bar{x} - \mu}{s/\sqrt{n}} \sim t_{n-1}$}

\newpage
%%%%%%%%%%%%%%%%%%%%%%%%%%%%%%%%%%%%%%%%%%%%%%%%%%%%%%%%%%%%%%%%%%%%%%%%%%%%%%%%%%%%%%%%%%%%%%%%%%%%%%%%%%%%%%%%%%%%%%%%%%%%%%%%%%%%%%%%%%%%%%%%%%%
\section{Tests}
Die Formeln für die Annahmebereiche stehen auf dem Angabeblatt.

\subsection{Leite den Annahmebereich für den ein-/zweiseitigen t -Test her.}

\subsection{$t$-Test/ANOVA/Binomialtest/$\chi^2$-Anpassungstest/$\chi^2$-Unabhängigkeitstest auf bestimmter Stichprobe durchführen.
Siehe PS-Beispiele.}

\subsection{Wie funktioniert der Kolmogorow-Smirnow-Test? Was ist die Nullhypothese?}

\newpage
%%%%%%%%%%%%%%%%%%%%%%%%%%%%%%%%%%%%%%%%%%%%%%%%%%%%%%%%%%%%%%%%%%%%%%%%%%%%%%%%%%%%%%%%%%%%%%%%%%%%%%%%%%%%%%%%%%%%%%%%%%%%%%%%%%%%%%%%%%%%%%%%%%%
\section{Simulation}
\subsection{Berechne Zufallszahlen mit dem additiven Kongruenzgenerator. Parameter und Seed sind gegeben, die Formel nicht.}

\subsection{Formuliere die Methode der inversen Transformation für nicht-gleichverteilte Zufallsvariablen und beweise sie.}

\subsection{Methode der inversen Transformation anwenden für einfache Verteilung.}

\subsection{Was ist eine Monte-Carlo-Methode?}

\subsection{Erkläre kurz die Vorgangsweise bei der Simulation zeitlicher Prozesse.}


\end{document}
